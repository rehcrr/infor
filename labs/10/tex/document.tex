\documentclass[12pt, a4paper]{article}
\usepackage[T2A]{fontenc}
\usepackage[utf8]{inputenc}
\usepackage[english,russian]{babel}
\usepackage{indentfirst}
\usepackage{graphicx}
\graphicspath{ {F:\image1} }
\begin{document}
\includegraphics{image1.png}\\	
такой, что $f'(\xi)=0$.(При этом, в силу условия 3, в котором\\
говорится о значениях функции в концевых точках проме-\\
жутка, следует рассматривать лишь функции, определен-\\
ные на отрезках.)
\par
Функция $f(x)$, определенная на отрезке [0, 1] и равная x,\\
и 3, но не удовлетворяет условию 1 (рис. 52).
\par
Функция $f(x)=|x|$, $x \in [0; 1]$ удовлетворяет условиям 1\\
и 3, но не удовлетворяет условию 2 (рис. 53).
\par
Наконец, функция $f(x)=x$, $x \in [-1; 1]$ удовлетворяет\\
условиям 1 и 2, но не удовлетворяет условию 3 (см. рис. 50).
\par
Для всех этих функций не существует точки, в которой\\
их производная обращалась бы в нуль.
\par
Обратим внимание на то, что по условиям теоремы Ролля\\
отрезок [a, b] может содержать точки, в которых функция\\
имеет определенного знака бесконечную производную, т.е.\\
в которых $\displaystyle{\lim_{\Delta x \to 0}} \frac{\Delta y}{\Delta x} = +\infty$ или $\displaystyle{\lim_{\Delta x \to 0}} \frac{\Delta y}{\Delta x} = -\infty$. Это требование\\
нельзя ослабить, заменив его условием $\displaystyle{\lim_{\Delta x \to 0}} \frac{\Delta y}{\Delta x} = \infty$. На-\\
пример, для функции $f(x) = \sqrt{|x|}$, $-1 \leq x \leq 1$, не существует\\
точки  $\xi = [-1, +1]$, в которой производная этой функции об-\\
ращалась бы в нуль. Вместе с тем функция $f(x) = \sqrt{|x|}$ удов-\\
летворяет всем условиям теоремы Ролля на отрезке [-1, 1],\\
за исключением того, что в точке х = 0 эта функция не имеет\\
ни конечной, ни определенного знака бесконечной произ-\\
водной (рис. 54).
\par
В самом деле, для этой точки $\displaystyle{\lim_{\Delta x \to 0}} \frac{\Delta y}{\Delta x} = \infty$, причем этот\\
предел не является бесконечностью определенного знака.
\setcounter{page}{317}
\end{document}