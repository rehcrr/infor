\documentclass[14pt]{extarticle}
\usepackage[T2A]{fontenc}
\usepackage[utf8]{inputenc}
\usepackage[english,russian]{babel}
\usepackage{indentfirst}
\usepackage{graphicx}
\usepackage{ragged2e}
\usepackage{changepage}
\usepackage{graphicx}
\usepackage{geometry}
\geometry{a4paper, left=3.5cm, right=0cm, top=3cm, bottom=0cm}
\begin{document}
\includegraphics[width=0.76\textwidth]{image1.png}\\
такой,\, что\, $f'(\xi)=0$.(При\, этом,\, в силу\, условия\, 3,\,\, в котором\\
говорится\, о\,\,  значениях\,\, функции\, в\,\,  концевых\,\, точках\,\, проме-\\
жутка,\,\,\, следует\,\,\, рассматривать\,\,\, лишь\,\, функции,\,\,\, определен-\\
ные на отрезках.)
\par
Функция\, $f(x)$,\, определенная\, на\, отрезке [0, 1]\, и\, равная x,\\
если\, $0 \leq x\! < 1$, и 0,\, если $x = 1$,\, удовлетворяет условиям 2 и 3,\\
и 3, но не удовлетворяет условию 1 (рис. 52).
\par
Функция~~\,\, $f(x)=|x|$, $x \in [0; 1]$~ удовлетворяет~ условиям 1\\
и 3, но не удовлетворяет условию 2 (рис. 53).
\par
Наконец,~\,\,\, функция~\,\,\, $f(x)=x$, $x \in [-1; 1]$~\,\, удовлетворяет\\
условиям 1~ и~ 2, но не\, удовлетворяет\, условию\, 3 (см. рис. 50).
\par
Для~ всех\,\, этих~ функций\space не~ существует\, точки,\, в которой\\
их производная обращалась бы в нуль.
\par
Обратим внимание\, на то, что по условиям теоремы\space\space Ролля\\
отрезок\, [a, b]\, может\space\space содержать\space\space точки,\space\space в\space\space которых\space\space функция\\
имеет\space\space определенного\space\space знака\space\space бесконечную\space\space\space производную,\space\, т.е.\\
в которых $\displaystyle{\lim_{\Delta x \to 0}} \frac{\Delta y}{\Delta x} = +\infty$ или $\displaystyle{\lim_{\Delta x \to 0}} \frac{\Delta y}{\Delta x} = -\infty$. Это требование\\
нельзя\,\,\, ослабить,\space\space заменив\space\space его\space\space условием\space\, $\displaystyle{\lim_{\Delta x \to 0}} \frac{\Delta y}{\Delta x} = \infty$.\,\, На-\\
пример,\, для функции $f(x) = \sqrt{|x|}$, $-1 \leq x \leq 1$,\, не существует\\
точки\,  $\xi = [-1, +1]$,\, в которой\, производная\, этой\, функции об-\\
ращалась\, бы в нуль. Вместе с тем функция $f(x) = \sqrt{|x|}$ удов-\\
летворяет\, всем\, условиям\,\, теоремы\,\, Ролля\,\, на\,\, отрезке\,\, [-1, 1],\\
за\, исключением того, что в точке х = 0 эта функция не имеет\\
ни\,\,\, конечной, ни\,\,\, определенного\,\,\, знака\,\,\, бесконечной\,\,\,\, произ-\\
водной (рис. 54).
\par
В самом деле, для\, этой\,\, точки\, $\displaystyle{\lim_{\Delta x \to 0}} \frac{\Delta y}{\Delta x} = \infty$, причем\,\, этот\\
предел не является бесконечностью определенного знака.\\
\begin{center}
	\hspace{-83pt} \line(1,0){70}
\end{center}
~~~~~~~~~~~~~~~~~~~~~~~~~~~~~~~~~~~~~~~~~~ 317
\thispagestyle{empty}
\end{document}